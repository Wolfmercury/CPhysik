\documentclass[12pt,a4paper,titlepage]{scrreprt}
\usepackage[utf8]{inputenc}
%\usepackage{amsmath}	%sonst für Mathe zustädnig
\usepackage{mathpazo} 	%für das |R etc. Mathemodus
\usepackage{commath}
\usepackage{IEEEtrantools}
\usepackage{siunitx}
%textart aendern
\usepackage[varg]{txfonts}
\usepackage{enumitem}
\usepackage{graphicx}
\usepackage{caption}
\usepackage{empheq}    % lädt »mathtools«, welches wiederum »amsmath« lädt
\usepackage{subcaption}
\usepackage{float} %erzwingt ort der Figure
\usepackage{amsfonts}
\usepackage{amssymb}
\usepackage{multirow}	%Multirow
\usepackage{graphicx}
\usepackage{xcolor}
\usepackage{cancel}
\usepackage{arydshln}	%für Gepunktete Linien bei tabellen
\usepackage[ngerman]{babel}
\usepackage[autostyle=true]{csquotes}
\usepackage{hyperref}
\graphicspath{{figs/}}
\usepackage{edv_makros}
\author{Kilian Kranz \& Achim Bruchertseifer}
\title{Treibhauseffekt}
\subtitle{Übungsblatt 3 zur Vorlesung \glqq Computerphysik\grqq~ SS 2018}
\begin{document}
	\maketitle
	\setcounter{chapter}{3}

	\section{Problemstellung}
	
	\begin{align}
	\epsilon(T)=&~\dfrac{1}{Z}\underbrace{\int\limits_0^\infty~\rho\left(\nu,T\right)\cdot\left[1-f(\nu\right]~\dif\nu}_{\text{numerischer Teil}}~,~~~~Z=\underbrace{\int\limits_0^\infty~\rho\left(\nu,T\right)~\dif\nu}_{\text{analytischer Teil}}
	\end{align}
	
	\section{Lösung des Problems}
	\subsection{Lösung des analytischen Integrals}
	Über eine \textsc{Taylor}entwicklung über den Ausdruck $\dfrac{1}{\euler^x-1}$ ergibt sich mittels Partieller Integration:
	\begin{align}
		Z=&~\sum_{m}^{\infty}\dfrac{48\pi~k^4}{c^3~h^3~m^4}~T^4~~.
	\end{align}
	Dies kann nun bis zu einer bestimmten Genauigkeit implementiert werden.
	
	\section{Ergebnis}
	

	
	\section{Anhang}

	
\end{document}